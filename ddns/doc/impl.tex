\section{Implementation}

\subsection{DDNS Daemon}
The DDNS daemon 
performs look-ups and stores of DDNS records
by interacting with a Dhash server.

\subsection{DDNS Server}

We plan to set up a DDNS server, running DDNS daemons,
to serve some information for personal domains.
The only difficult part about this is starting the signature
verification tree, since the root servers don't store public
key information. (For the purposes of our simulations,
we will invent a ``well-known'' public key for the DNS root.)
To bootstrap ourselves,
the NS record for a given domain to be served
by Chord will contain a SHA1 cryptographic hash of the
public key for that domain.
For example, the NS record for {\tt uvxyz.org}
is {\tt 551cdc73ac64e15193eeab47bf0f3fab8c118fd6
.chorddns.net}
with a glue record translating this fake name to a real IP address.
When a DDNS daemon receives a request for {\tt host.dom.uvxyz.org},
it queries the root server to find the public key hash,
performs a DHash lookup to find the DDNS record for {\tt host.dom.uvxyz.org},
and then verifies it starting with the {\tt uvxyz.org} root.

\subsection{Unresolved Issues}

We have been assuming that DNS queries can be predicted in advance,
and responses stored in the network.
On the real internet, DNS data is sometimes dynamically generated in response
to the particular query.  
For example, some firewalls will respond with mail exchanger (MX) records
for nonexistant host names, to avoid publishing internal host names.
For example, looking up {\tt 1234abcd.cs.bell-labs.com}'s MX record
succeeds even though {\tt 1234abcd.cs.bell-labs.com} does not exist.
As another more insidious example, content distribution services like 
Akamai~\cite{akamai}
route web requests to nearby servers by modifying DNS responses in response
to the location of the client in the network.

The MX record problem can be solved by walking up the hierarchy.
The Akamai problem is harder, but Akamai is breaking one of the 
central tenets of DNS --- the same answers to everyone --- as a
result we do not feel responsible to solve this problem.
