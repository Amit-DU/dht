\section{Introduction}

The current DNS~\cite{dns, dns-concept:rfc} relies heavily on its
thirteen root servers, which are responsbile for providing
information about the root of the hierarchy and many top-level domains.
The Internet Software Consortium, which runs one of the root servers,
reported a load of 272 million queries per day in 2001~\cite{isc-root}.
Recent reports~\cite{cert} have indicated that, just by running 
older versions of the name resolution software (eg. ISC BIND), 
name servers can set themselves up for denial-of-service attacks 
(causing {\tt named} daemons to crash or go into infinite loop).
If the root name servers were to be compromised, the Internet can 
come to a halt especially when as much as 18\% of DNS lookups 
can end up at the root servers. Breaking away from the heirarchical 
pratice of serving DNS data can lessen impact of load and the severity 
of security compromises faced by the root servers.

%Leslie Lamport's supposed characterization of a
%distributed system (as one in which you can't get work
%done because of a machine you have never heard of is down)
%is all too true in the current implementation of {DNS}.
Jung {\it et al.}~\cite{dnscache:sigcommimw01} found that approximately
35\% of DNS queries never receive an answer or receive
a negative answer. Many of these failures are due to 
improperly configured name servers or incorrect {\tt NS}-records).
For example, {\tt NS}-records that point to non-existent 
or non-authoritative name servers can cause lookups to fail
or succeed only some of the times (if the server happens
to have the record cached). 
Configuring a name server can be an administrative nightmare.
Poor network conditions are also to blame for failures in DNS lookups.
Potential problems~\cite{dns-bind} that often arise include: 
\begin{itemize}
\item Forgetting to increment the serial number when updating the 
zone's database file in the primary name server (causing slave name 
servers not to reload the new zone data).
\item Forgetting to signal the primary name server 
(to reload new data) after 
changing the configuration or database file. 
\item Errors in the configuration or database file. 
(Our system will eliminate errors in the config file, 
and the ones in the database relating to {\tt NS}-records.) 
\item Missing cache data either because the adminstrator 
forgot to install the cache file or accidentally deleted it. 
This will cause all lookups outside 
of the name server's authoritative zone to fail. 
\item Loss of network connectivity to a name server 
can cause some DNS lookups to fail if the name server is
the only one with the zone data. 
\item A slave server can fail to load zone data 
if it lost connectivity to the master server due to 
network failure, has the wrong IP address for
the master server in the config file, or if there is 
a syntax error in the zone data file on the master server.
%-missing or incorrect subdomain delegation.\\
\end{itemize}

%Such failures can go undetected because of the hierarchical
%nature of DNS: lookups within an administrative domain might
%succeed but that domain is unknowningly cut off from the rest
%of the hierarchy, so that lookups from outside will fail.

We propose a distributed domain name service, DDNS,
using a robust peer-to-peer lookup service such as Chord~\cite{chord:sigcomm}.
Chord balances load as a side effect of its lookup and storage 
algorithms, thus avoiding such hot spots.
DDNS acts as a replacement to name servers, so there is no need 
to manage such servers, thereby eliminating all of 
the configuration problems and incorrect name server mapping 
({\tt NS}-records) mentioned above. 
The Chord protocol is robust against server failures 
(host down or lost connection),
hence DDNS can reduce errors due to poor network connectivity as well.

Every DNS lookup is fed directly to a group of servers that form
the Chord ring. Treat each lookup as flat.
 
Even-handed discussion.\\
Distributing DNS data in a peer-to-peer network does have some
disadvantages.In particular, we cannot easily dynamically 
generate DNS responses. We discuss various solutions to 
this problem in the implementation section.
Separating the mechanism to lookup DNS information from the one 
in which to verify it can increase performance and flexibility 
in designing the name resolution system. Although it can open up 
security holes. More cons...


%**How to control caching on others name servers...** --obsolete b/c we're 
%replacing name servers.


