\section{Introduction and Related Work}

In the beginnings of the Internet, host names were kept
in a centrally-administered text file, {\tt hosts.txt},
maintained at the SRI Network Information Center.
By the early 1980s the host database had become too large
to disseminate in a cost-effective manner.
In response, Mockapetris and others began the design and
implementation of a distributed database that we now know
as the Internet domain name system (DNS).
DNS started to replace {\tt hosts.txt} in 1985 and has remained
substantially the same since then~\cite{dns-concept:rfc, dns}.

Looking back at DNS in 1988, Mockapetris and Dunlap~\cite{dns}
listed what they believed to be the surprises, successes,
and shortcomings of the system.  Of the six successes,
three (variable depth hierarchy, organizational structure
of names, and mail address cooperation) relate directly 
to the adoption of an administrative hierarchy for the names.
The administrative hierarchy of DNS is reflected in the 
structure of DNS servers: in typical usage, an
entity is responsible not only for maintaining name information
about its hosts
but also for the serving that information.

The fact that service structure mirrored administrative hierachy
provided a modicum of authentication for the returned data.
DNS clients authenticate the responses to their queries by
looking at the source IP address on the response:
clients know and trust the IP addresses of the root servers, and then
query the root servers to find the IP addresses of servers
responsible for particular sub-hierarchies, which in turn are
trusted, and so on.
Unfortunately, IP addresses can be forged and thus it is possible
for malicious people to impersonate DNS servers.
In response to concerns about this and
other attacks, the DNS Security Extensions~\cite{dnssec:rfc}
(DNSSEC) were developed in the late 1990s.
DNSSEC provides a stronger mechanism for clients to verify that the
records they retreive are authentic.
Although DNSSEC is not yet in widespread use, its specification
has stabilized, and it is supported by a variety of name server
implementations ({\em e.g.}, the Internet Software Consortium's
BIND name server software).

DNSSEC effectively separates the authentication of data from the
service of that data.  This observation enables the exploration of alternate service 
structures to achieve desirable properties not possible
with conventional DNS.  In this paper, we explore one
alternate service structure based on Chord~\cite{chord:sigcomm},
a peer-to-peer lookup service.

Rethinking the service structure allows us to address
some of the current shortcomings in the current DNS.
The most obvious one is that it requires 
significant expertise to administer.
In their book on running DNS servers using BIND,
Albitz and Liu~\cite{dns-bind} note that many of the most
common name server problems are configuration errors.
Name servers are difficult and time-consuming to administer;
ordinary people typically rely on ISPs to serve their name data.
Our approach solves this problem by separating service from
authority --- clients can enter their data into the internet-wide
Chord storage ring and not worry about needing an ISP to be
online and to have properly configured its name server.

DNS performance studies have confirmed this folklore.
In 1992, Danzig {\it et al.}~\cite{dnsroot:sigcomm92} found that
most DNS traffic was caused by misconfiguration and faulty implementation
of the name servers.
They also found that one third of the 
traffic that traversed the NSFNet was directed to one of 
the seven root name servers. 
In 2000, Jung {\it et al.}~\cite{dnscache:sigcommimw01}
found that approximately 35\% of DNS queries never receive
an answer or receive a negative answer, and attributed
many of these failures to 
improperly configured name servers or incorrect name server (NS) records.
The study also reported that as much as 18\% of DNS traffic is destined
for the root servers. 
CERT~\cite{cert} has documented serveral name server
denial-of-service attacks taking advantage of buffer overflows
and other vulnerabilities in the server implementation.

Serving DNS data over Chord eliminates the need to have
every system administrator be an expert in running name servers.
It provides better load balance, since the concept of root server
is eliminated completely.
Finally, it provides robustness against denial-of-service attacks
since disabling even a sizable number of hosts in the 
Chord network will not prevent data from being served.
