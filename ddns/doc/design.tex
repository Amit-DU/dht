\section{Design}

DDNS makes use of
DHash~\cite{cfs:sosp01}, a distributed hash table
built on top of Chord~\cite{chord:sigcomm}, to store
and retrieve resource records (RRs).
Dhash has the following two nice properties:

\begin{list}{$\bullet$}{}
\item {\em Load balance}.
If a particular record is looked up $n$ times
in succession by different hosts (where $n$ can be very large),
then with high probability each server sends
the record to another server only $\log{m}$ times, 
where $m$ is the number of servers. Notice that
the number of times a record is transferred is 
{\it independent} of the how popular it is.

\item {\em Robustness}.
As servers are disconnected from and reconnected to the network,
DHash automatically moves data so that it is always on a fixed
number of available replicas (typically six).
Further, a very large number of servers have to
all be disconnected simultaneously before data loss
becomes apparent.
\end{list}

\subsection{Resource Records}

All RRs for a domain name are stored 
as a single record (hereafter referred to as a DDNS~record) 
indexed by the domain name.
All transactions are done on the granularity
of DDNS records. The reason for grouping RRs is because it 1)
guarantees that all RRs for a domain name will reside on
the same server, 2) complements the verification scheme
of DNSSEC (explained below), and 3) results in a simple
and functional design altogether. We feel that this decision 
will not degrade performance, since RRs are typically 
very small, a DDNS~record should only take one RPC to transmit.

\subsection{Stores}

To store a DDNS record, the owner of that record 
assembles all the RRs associated with the domain name,
and sends a {\tt ddns-store(name, DDNS~record)} message to DDNS. 
DDNS calculates the hash of the domain name, and sends
a {\tt dhash-store(hash, data)} RPC to DHash, using the
DDNS record as the data.

\subsection{Look-ups}

To perform a DNS look-up, a client will send a 
{\tt ddns-lookup(name)} message to {DDNS}.
DDNS 
calculates the hash of the domain name, sends a
{\tt dhash-lookup(hash)} RPC to DHash. Upon receiving
the response, DDNS returns the corresponding DDNS record to the 
client or report an error of the domain name does 
not exist.

\subsection{DDNS Record Verification}

Since records are being served from arbitrary hosts rather
than the hosts responsible for the domain, 
we use the DNS Security Extensions (DNSSEC)~\cite{dnssec-rfc} to
verify all records before returning them. DNSSEC stores the 
signature of the owner of the RRSet~\cite{dnscla-rfc} in a 
SIG RR.
% We use one {\tt SIG RR} to carry the owner's signature
% of the entire DDNS record, since we always transmit data on
% the granularity of DDNS records.
