\section{Related Work}

To the best of our knowledge, there have not
been any systems that distribute DNS data away
from the hosts responsible for it.
Several DNS performance
studies have indicated shortcomings 
of traditional DNS, and research on cooperative 
systems has reported promising results using 
peer-to-peer storage systems. In addition, research on
DNS security has armed us with schemes to authenticate
DNS records served by hosts that are not authorized for them.
 
\subsection{DNS Performance Studies}

Danzig~{\it et al.}~\cite{dnsroot:sigcomm92} performed 
a study of DNS traffic at a root name server in 
1992. The study concluded that most of the DNS traffic 
was caused by misconfiguration and faulty implementation
in the name servers. The most dominant error is looping caused
by receiving a response to a query that contains no answer
accompanied by a list of other name servers to contact, which
includes the sender of this response. Naive name servers 
will resend the queries to the same server, generating a loop.
The study also reported that one third of the traffic 
that traversed the NSFNet was directed to one of (in 1992)
seven root name servers. 

Jung~{\it et al.}~\cite{dnscache:sigcommimw01} studied 
client-perceived DNS performance in 2000 and 2001.
They report that 23\% of DNS lookups fail to elicit
any response, partially due to loops in name server resolution.
13\% of lookups result in a negative response--many of which 
are caused by {\tt NS} records that point to non-existent 
or inappropriate hosts.

DDNS attempts to alleviate the poorly-configured
name server problem by replacing name server topology.
It does not suffer from looping, because it does not depend
on information in the DNS records (which are often erroneous)
to resolve queries. Instead, DDNS harnesses a group of machines
to form a robust storage/lookup system for DNS data. 

Jung~{\it et al.} also found that {\tt NS}-record caching
is critical to DNS scalability since it reduces load on 
root servers. Our system directly tackles this problem by 
eliminating the server hierarchy entirely.
Every server participating in DDNS has roughly
the same load due to the use of consistent hashing in Chord.

\subsection{Cooperative Systems}

Recent file systems such as CFS~\cite{cfs:sosp01} and 
PAST~\cite{past:sosp} have demonstrated that cooperative 
peer-to-peer storage systems 
are highly scalable and resilient to failures. We attempt
to show that DNS can also acheive such scalability 
and fault tolerance by operating in a peer-to-peer fashion.

%The study on cooperative Web proxy caching~\cite{coop-web-cache:sosp99}
%has shown that, based on simulations of using traces of
%client web access patterns, cooperative caching is a beneficial
%model to transmit small data with reasonable ...

\subsection{DNS Security}

The DNS Security Extensions (DNSSEC)~\cite{dnssec:rfc} 
provides data integrity and authentication to security aware 
resolvers and applications. An owner of a domain name uses
cryptographic digital signatures to sign each RRSet~\cite{dns-cla:rfc}
and store the signature in the SIG resource record. 
Authentication is done by retrieving the public key of the
owner and verifying that data in the RRSet check with the 
associated SIG RR. Resolvers can retrieve public keys by 
requesting the KEY resource record, which also has an associated
SIG RR. KEY RRSets are signed by a higher level zone.

SDSI

DDNS adopts the RRSet authentication mechanisms from DNSSEC,
and the key distribution scheme from SDSI.
