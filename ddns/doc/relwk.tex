\section{Related Work}

To the best of our knowledge, there have not
been any systems that distribute DNS data away
from the hosts responsible for it.
Several DNS performance
studies have indicated shortcomings 
of traditional DNS, and research on cooperative 
systems has reported promising results using 
peer-to-peer storage systems.
 
\subsection{DNS Performance Studies}

Danzig~{\it et al.}~\cite{dnsroot:sigcomm92} performed 
a study of DNS traffic at a root name server in 
1992. The study concluded that most of the DNS traffic 
was caused by misconfiguration and faulty implementation.
The study also reported that one third of the traffic 
that traversed the NSFNet was directed to one of (in 1992)
seven root name servers. 

Jung~{\it et al.}~\cite{dnscache:sigcommimw01} studied 
client-perceived DNS performance in 2000 and 2001.
They report that 23\% of DNS lookups fail to elicit
any response, partially due to loops in name server resolution.
13\% of lookups result in a negative response.

DDNS attempts to alleviate the poorly-configured
name server problem by replacing name server topology.
Instead, DDNS harnesses a group of machines
to form a robust storage system for DNS data.

Jung~{\it et al.} also found that {\tt NS}-record caching
is critical to DNS scalability since it reduces load on 
root servers. Our system directly tackles this problem by 
eliminating the server hierarchy entirely.
Every server participating in DDNS has roughly
the same load due to the distributed property of Chord.

\subsection{Cooperative Systems}

Recent file systems such as CFS~\cite{cfs:sosp01} and 
PAST~\cite{past:sosp} have demonstrated that cooperative 
peer-to-peer storage systems 
are highly scalable and resilient to failures. We attempt
to show that DNS can also acheive such scalability 
and fault tolerance by operating in a peer-to-peer fashion.

%The study on cooperative Web proxy caching~\cite{coop-web-cache:sosp99}
%has shown that, based on simulations of using traces of
%client web access patterns, cooperative caching is a beneficial
%model to transmit small data with reasonable ...

\subsection{DNS Security}
