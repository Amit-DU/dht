\documentclass[letter,10pt]{article}

\usepackage{fullpage}
\usepackage{latexsym}
\usepackage{amsmath}

\addtolength{\topmargin}{+6mm}

\begin{document}

\begin{flushright}
Steven Gerding, sgerding@mit.edu \\
Jeremy Stribling, strib@mit.edu \\
6.829 Project Proposal \\
9/29/03
\end{flushright}

\begin{center}
{\bf Comparing Peer-to-Peer Overlays using the P2P Simulator}
\end{center}

{\bf Abstract:} We will compare several structured peer-to-peer overlay 
algorithms in a fair and balanced manner using the newly-developed p2p 
simulator, and provide analysis regarding their relative strengths and 
weaknesses.

{\bf Proposal:} There have been many recent studies comparing different
aspects of structured peer-to-peer overlays, however we feel these comparisons
either ignore the relative strengths of the algorithms by hiding them within
a black box~\cite{rrk-iptps03}, or focus on the more abstract aspects of 
peer-to-peer overlay design~\cite{gummadi-geometry} rather than evaluating 
the overall performance of complete implementations.  In contrast, we seek
to compare several peer-to-peer overlay algorithms in an environment that
is both fair and comprehensive.  We plan to use the p2p simulator, a tool 
recently developed by the Parallel and Distributed Operating Systems group at 
MIT, which allows different peer-to-peer algorithms to be rapidly implemented 
and tested within a common framework.  We will focus on the performance (in 
terms of lookup latency, storage overhead, and bandwidth usage) of
various algorithms when faced with differing network topologies, node arrival
and failure (churn) models, and non-transitive network links.  We will be 
examining and analyzing the strengths and weaknesses of Chord~\cite{Chord}, 
Kademlia~\cite{Kademlia}, and Tapestry~\cite{tapestry_jsac}, chosen to 
represent a cross-section of the current state of overlay network research.

{\bf Methodology:} As mentioned above, we will be using the p2p simulator
to perform our experiments.  It provides a simple C++ interface for 
implementing peer-to-peer algorithms.  Implementations for Chord and Kademlia 
already exist, and as part of this project we will be completing the Tapestry 
implementation.  A primary focus of our research will be testing the various 
algorithms under accurate representations of real world network conditions.  
We plan to use data for the ping times, failure rates, and non-transitive 
connectivity of the nodes in the PlanetLab testbed, collected over the past
several months~\cite{appweb}, to generate topologies for the simulator.
We will simulate each algorithm on these topologies, using a standardized
key-lookup workload, and measure and analyze the effects of varying the
network conditions on the performance and success rate of these lookups.

{\bf Schedule}:  We hope to have partial results from this project available 
for the IPTPS '04 deadline on November $10^{th}$, so our schedule is structured
as follows:

\begin{table}[h]
\centerline{
\begin{tabular}{|l|l|} 
\hline
Complete Tapestry implementation & Week of 10/13/03 \\
\hline
Parsing APP data / topology generation & Week of 10/20/03 \\
\hline
Run simulations on static topologies & Week of 10/27/03 \\
\hline
Run churn simulations & Week of 11/3/03 \\
\hline
Write analysis of data so far & 11/10/03 (IPTPS) \\
\hline
Run changing topology simulations & Week of 11/17/03 \\
\hline
Run non-transitive link simulations & Week of 11/24/03 \\
\hline
Make presentation slides & 12/4/03 \\
\hline
Write paper & 12/12/03 \\
\hline
\end{tabular}
}
\end{table}

{\bf Resources:} To complete our project, we will be using the following 
resources: \\
\begin{table}[h]
\centerline{
\begin{tabular}{ll} 
The p2p simulator source code & $\surd$ \\
A FreeBSD machine on which to run simulations & $\surd$ \\
\end{tabular}
}
\end{table}

{\bf Comments:} We welcome any feedback regarding this project.

\begin{footnotesize}
\bibliographystyle{acm}
\bibliography{proposal.bib}
\end{footnotesize}

\end{document}
