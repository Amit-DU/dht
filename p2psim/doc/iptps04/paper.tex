% Headers {{{1
\documentclass[letterpaper,twocolumn,10pt]{article}
\usepackage{url,graphicx,epsfig,alltt,fancybox}
\usepackage[english]{babel}
\usepackage{times}
\usepackage[colorlinks,linkcolor=blue,citecolor=blue,urlcolor=blue]{hyperref}

\title{\Large \bf Snore: the peer-to-peer simulator we were all waiting for}
\author{Zeeb and Zoob \\
{\em CSAIL, M.I.T., Cambridge, MA} \\
{\tt \{zeeb,zoob\}@csail.mit.edu}}
\date{}
\begin{document}
\maketitle

% 1}}}
% Abstract {{{1
\begin{abstract}
\end{abstract}

We want to dominate the world, so everyone should use Snore.

% 1}}}
% Introduction {{{1
\thispagestyle{empty}
\section{Introduction}
\label{Section:Introduction}

A large number of P2P protocols have been proposed in the literature.  There
is no unified framework for evaluating the performance of these protocols.
Deploying in real-world network can be complex and depending on the comparison
you wish to do, not even useful.

Framework for comparing P2P protocols.  Should allow for easy coding of
protocols and simple models for underlying topologies.  Preferably, allow for
pseudo code to be written, abstracting away all the intricacies of network
protocols, etc.

Why another simulator?  Why not NS?

Challenges: scaling.

Why is this a reasonable thing to do?  Not oversimplified?  Results correspond
with real life!

% 1}}}
% Related work {{{1
\section{Related work}
\label{Section:Related}

Other simulators suck.

% 1}}}
% Design considerations {{{1
\section{Design considerations}
\label{Section:Design}

Maybe this section is useless.  In fact, maybe this entire paper is useless.


% }}}1
% Implementation {{{1
\section{Implementation}
\label{Section:Implementation}

Which protocols supported.  How many lines of code for each compared to the
real thing?  How easy was it to take pseudocode and dump it in the simulator.

Russ' threads package allows for simple coding.

% }}}1
% Experiments {{{1
\section{Experiments}
\label{Section:Experiments}

Number of threads.  Number of nodes.  How many messages can we deliver to
nodes per second, or something like that.

Show that numbers that come out have some correspondence to the real world.
Topology models are OK, demonstrate with King data set.

Where is the bottleneck?

[Must hack simulator to reduce threads.]


% 1}}}
% Conclusion {{{1
\section{Conclusion}
\label{Section:Conclusion}

% 1}}}
% Acknowledgements {{{1
\section{Acknowledgements}
\label{Section:Acknowledgements}

\vfill

% 1}}}
% Bibliography {{{1
\pagebreak

% \begin{flushleft}
% % \bibliographystyle{alpha}
% % \bibliography{paper}
% \end{flushleft}

\end{document}
% 1}}}
% vim: set foldlevel=0 fen:
